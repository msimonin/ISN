\documentclass[a4paper,11pt,dvips]{article}
\usepackage{blum}
\usepackage{geometry}
\geometry{hmargin=2cm,vmargin=1cm}
\usepackage{hyperref}
\usepackage{verbatim}
\pagestyle{empty}
\begin{document}
 
\entete{Deuxi�me partie}{ProjetISN:12/13}

La premi�re partie avait pour but de vous familiariser avec l'environnement et d'�tudier deux algorithmes de g�n�ration 
de labyrinthes : al�atoires et fractals.\footnote{J'esp�re pouvoir voir bient�t de jolies figures en particulier sur les labyrinthes fractals
(autres que celles donn�es dans l'�nonc�)}.

Dans cette deuxi�me partie nous allons continuer notre cheminement dans le monde des labyrinthes\dots


  \subsection{Qualit� d'un labyrinthe}
    Les premiers labyrinthes construits ont des allures bien diff�rentes.
    \'Ecrire en python des programmes prenant en entr�e un labyrinthe et calculant : 
    \begin{enumerate}[$\bullet$]
     \item sa densit� : c'est � dire le ration cases noires / cases blanches,
     \item sa plus grande ligne droite,
     \item son nombre de cul-de-sacs,
     \item son nombre de carrefours.
    \end{enumerate}
  Les deux derniers points se calculeront plus facilement sur des labyrinthes dont le couloirs ne font pas plus d'une case de largeur, vous aurez des exemples dans le r�pertoire \texttt{qualite}.

  \subsection{D'autres algorithmes}

    Comme vous aurez pu le constater en parcourant les ressources � votre disposition sur le sujet, 
    il existe de nombreux algorithmes de g�n�ration de labyrinthes.\\
    L'algorithme de Prim en fait partie. \\
    Votre travail consiste �:
    \begin{enumerate}[$\bullet$]
     \item comprendre cet algorithme, c'est � dire � �tre capable � l'aide d'une feuille 
	    et d'un stylo (et d'une gomme !) de g�n�rer des labyrinthes de petites tailles,\\
     \item le comparer avec un autre algorithme de votre choix (autre que ceux vus pr�c�demment), \\
     \item le coder en python.
    \end{enumerate}


\end{document}
