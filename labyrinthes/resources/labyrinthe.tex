\documentclass[a4paper,12pt,dvips]{article}
\usepackage{blum}
\usepackage{geometry}
\geometry{hmargin=2cm,vmargin=1cm}
\usepackage{hyperref}
\pagestyle{empty}
\begin{document}
 
\entete{Dans un labyrinthe et autour \dots}{TS-ISN:12/13}

\subsection{Principaux objectifs}

  \begin{enumerate}[$\bullet$]
    \item Savoir d�finir et reconna�tre diff�rents types de labyrinthe : 
      \begin{itemize}
	\item parfait
	\item non parfait
      \end{itemize}.
  
    \item Savoir faire le lien labyrinthe $\leftrightarrow$ graphe. \\
	  C'est un arbre dans le cas d'un labyrinthe parfait.

    \item Savoir mettre en \oe uvre un algorithme de recherche de de la sortie dans un labyrinthe parfait.
	  Lien avec le parcours sur l'arbre.

    \item Comprendre l'algorithme de Prim de g�n�ration de labyrinthe parfait.
	  Proposer une am�lioration de l'algorithme pour am�liorer la \og qualit� \fg du labyrinthe

    \item L'algorithme de Prim dans un autre contexte $\leftrightarrow$ : recherche d'un arbre couvrant minimal. \\
	  Exemple d'un r�seau �lectrique.
  \end{enumerate}

\subsection{Modalit�s}

  \subsubsection{S�ance 1}
    \begin{itemize}
      \item Des exemples de labyrinthes (parfaits ou non),
      \item Un exemple de g�n�ration : al�atoire $\diamond$
      \item Graphes correspondants,
      \item Parcours ?
    \end{itemize}
    
  \subsubsection{S�ance 2}
    Calcul de qualit� d'un labyrinthe parfait.
     \begin{itemize}
      \item nombre de murs, $\star$
      \item Plus longue ligne droite, $\star$
      \item Nombre de virages\dots $\star$
\item Nombre de cul-de-sac. $\star$
     \end{itemize}

  \subsubsection{S�ance 3}
      Suite qualit�.
      Algorithme de Prim
  \subsubsection{S�ance 4}
      Algorithme de Prim am�lior�
  \subsubsection{S�ance 5}
    Application � la recherche d'un arbre couvrant minimal (ex d'un r�seau �lectrique)

\subsection{Resources}
\begin{tabular}{ll}
    
   culturelle 	& \url{http://fr.wikipedia.org/wiki/Labyrinthe}   \\

   informatique & \\
   \multicolumn{2}{l}{\url{http://fr.wikipedia.org/wiki/Mod\%C3\%A9lisation_math\%C3\%A9matique_d\%27un_labyrinthe}} \\

		& Tangente \no 57 \\
	    
		& Tangente Hors S�rie \no 12 \\

\end{tabular}
\end{document}



